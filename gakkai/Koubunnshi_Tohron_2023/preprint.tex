\documentclass[uplatex,dvipdfmx,a4paper,10pt]{jsarticle}

\usepackage{amsmath,amsthm,amssymb}
\usepackage[dvipdfmx]{graphicx}
\usepackage{bm}
%
\usepackage{multirow}
\usepackage{wrapfig}

\usepackage{geometry}
\geometry{left=25truemm, right=25truemm, top=25truemm, bottom=25truemm}

%
\abovecaptionskip=-1pt
%\belowcaptionskip=-1pt
%
\renewcommand{\baselinestretch}{0.84} %全体の行間調整
\renewcommand{\figurename}{Fig.}
\renewcommand{\tablename}{Tab.}
%
\makeatletter 
\def\section{\@startsection {section}{1}{\z@}{1.5 ex plus 2ex minus -.2ex}{0.5 ex plus .2ex}{\large\bf}}
\def\subsection{\@startsection{subsection}{2}{\z@}{0.2\Cvs \@plus.5\Cdp \@minus.2\Cdp}{0.1\Cvs \@plus.3\Cdp}{\reset@font\normalsize\bfseries}}
\makeatother 
%

\renewcommand{\thefootnote}{\fnsymbol{footnote}}

\pagestyle{empty}

\graphicspath{{../../figures//}}

\begin{document}

%%%%%%
% はじめに
%%%%%%
\begin{center}
{\Large \textgt{ランダムな接続性を有するネットワークポリマーの緩和挙動}}
\end{center}

\begin{flushright}
東亞合成 ${}^\circ$佐々木裕
\end{flushright}

% \renewcommand{\thefootnote}{\fnsymbol{footnote}}
\footnote[0]{
{\bf Relaxation Characteristics of Network Polymers with random connectivity using Molecular Dynamics Simulations} \\
\underline{Hiroshi SASAKI} (Toagosei Co., Ltd. 8, Showa-Cho, Minato-ku, NAGOYA 455-0026, JAPAN)\\
Tel: +81-52-611-9923, e-mail: hiroshi\_sasaki$@$mail.toagosei.co.jp
}

\vspace{-3mm}
\section{はじめに}

\subsection{ネットワークポリマー研究の深化と新規材料への展開}

一分子中に複数の反応性官能基を有する反応性オリゴマーを用いた硬化型樹脂は、接着剤、塗料等のさまざまな応用分野に展開されている。
これは、光や熱の外的な刺激により迅速にネットワーク構造を形成し、優れた機械的特性を示す。

Curable resins using reactive oligomers with multiple reactive functional groups in a single molecule have been deployed in a variety of applications such as adhesives and paints.
They rapidly form network structures upon external stimuli of light or heat, and exhibit excellent mechanical properties.

構造材料の軽量化につながる新たな複合材料の開発において、接着接合技術は重要なキーポイントであり近年注目を集めている。
その際、接着剤として使用されるネットワークポリマーの必要特性は、高い一次機械的特性だけでなく、長期使用における破壊や疲労に対する耐久性である。

Adhesive bonding technology is an important key point in the development of new composite materials that lead to weight reduction of structural materials and has attracted much attention in recent years.
In such cases, the properties required of network polymers used as adhesives are not only high primary mechanical properties, but also durability against fracture and fatigue in long-term use.


材料としての耐久性を保証するためには、さまざまな変形率で繰り返し変形させる疲労試験に耐えることも重要である。
この場合、適切な緩和時間で回復する可逆的なメカニズムに基づく強靭化メカニズムが必要となる。
このコンセプトに基づき、脆性破壊につながりやすい硬い機械的特性ではなく、しなやかな強度と延性を兼ね備えたネットワークポリマーが研究されている。

To guarantee durability as a material, it is also important to withstand fatigue tests in which the material is repeatedly deformed at various deformation rates.
In this case, a toughening mechanism based on a reversible mechanism that recovers with an appropriate relaxation time is needed.
Based on this concept, network polymers that combine supple strength and ductility, rather than hard mechanical properties that tend to lead to brittle fracture, have been studied.



\subsection{力学的ヒステリシスの重要性}

高分子材料の耐久性を保証するためには、機械的性質の発現機構とその劣化機構の解明が望まれる。

In order to guarantee the durability of polymeric materials, it is desirable to elucidate the mechanism of mechanical property development and its degradation mechanism.

破壊工学の概念を端的に表現すると、「システムに欠陥が存在することを前提とした耐久性の評価」である。
破壊挙動は、材料の脆性破壊を仮定した「グリフィス理論」における亀裂進展に伴うエネルギー放出率$G_c$と、$J$積分により非線形領域に拡張された$J_c$で議論され、これらの値が靭性の指標とされている。
しかし、破壊時の変形が極めて大きいソフトマター材料への適用には注意が必要である。

The concept of fracture engineering can be expressed simply as "the evaluation of durability based on the assumption that defects exist in a system.
Fracture behavior is discussed in terms of $G_c$, the rate of energy release associated with crack propagation in "Griffith Theory," which assumes brittle fracture of materials, and $J_c$, which is extended to the nonlinear region by $J$ integration, and these values are used as indicators of toughness.
However, caution should be exercised in their application to soft-matter materials, which are subject to extremely large deformation at fracture.

Andrews は、応力 - ひずみ関係における力学的ヒステリシスに着目し、ヒステリシスロスの存在により亀裂進展に伴うエネルギー開放量が減少し、結果として亀裂の進展が抑制されるモデルを提案した~\cite{Andrews1977}。
実際、上述のゴムにおけるフィラーの効果~\cite{Igarashi2013}や DN ゲルにおける犠牲結合~\cite{Gong2010}における大きなヒステリシスの存在は、この高靭性メカニズムの考え方に合致する。

Focusing on mechanical hysteresis in the stress-strain relationship, Andrews proposed a model in which the presence of hysteresis loss reduces the energy release associated with crack growth and consequently suppresses crack growth~\cite{Andrews 1977}.
Indeed, the presence of large hysteresis in the effect of filler in rubber~\cite{Igarashi2013} and sacrificial bonding in DN gels~\cite{Gong2010} is consistent with this idea of a high-toughness mechanism.

これらの例はすべて、分子鎖の描写よりもわずかに大きいメソスケール領域にあたり、この比較的大きなスケールでの挙動は、一般に長い緩和時間をもたらす。
ヒステリシス挙動はこのスケールでのみ起こるのだろうか?
われわれは、機械的ヒステリシスはより微視的な分子鎖の描像でも起こり、その結果、速い周期の刺激に対する破壊抵抗性が向上するのではないかと考えている。

All of these examples are in the mesoscale region, which is slightly larger than the description of the molecular chain, and behavior at this relatively large scale generally leads to long relaxation times.
Does hysteresis behavior occur only at this scale?
We hypothesize that mechanical hysteresis may also occur in more microscopic depictions of molecular chains, resulting in increased resistance to breakdown against fast-periodic stimuli.



\subsection{破壊にたいする粘弾性効果}
(破壊にたいする粘弾性効果)
破壊工学の考え方を端的に表せば、「系中に欠損が存在することを前提にした耐久性の評価」ということになろう。
脆性破壊する材料を想定した「Griffith 理論」での亀裂進展に伴うエネルギー開放率 $G_c$、さらには、$J$ 積分により非線形領域へ拡張された $J_c$ により破壊挙動が議論され、これらの値が靭性の指標となるとされている。
しかしながら、破断時の変形がけた違いに大きいソフトマター系の材料への適応には注意が必要である。

\subsection{疲労に対しての可逆性の重要性}

一般に、破壊試験による材料の強度評価は任意の変形速度での一回の変形挙動で評価されるため、ヒステリシスの回復挙動の遅速はあまり問題にならない。
しかしながら、材料としての耐久性を保証するためには、多様な変形速度での繰り返し変形を行う疲労試験に対する耐久性も重要である。
この場合、適正な緩和時間で回復する可逆的なメカニズムに基づく強靭化機構が必要となる。

\subsection{目指すもの}




\subsection{本検討内容}
(本検討内容)
ソフトマターの構造材料への展開を標語的に言えば、「脆性破壊を伴いがちな剛直性から、設計された延性に基づく高耐久性を示す『しなやかな強さ』へのパラダイムシフト」となるであろう。
この設計された延性に必要な要件を明確にすることが本研究の目的である。

本報告では、先行研究である Everaers らの方法~\cite{Everaers1999} に従った規則構造を有するネットワークの分子動力学(MD)シミュレーションによりそのゴム弾性挙動の変形速度依存性と緩和時間との関連について検討を行い、時間温度換算則が成り立つような線形粘弾性の枠組みでのヒステリシスを考察した。







ネットワークポリマー研究の深化と新規材料への展開


\subsection{ネットワークポリマーの破壊}

旧知の材料であるゴムの大きな破壊靭性の由来については、 ヒステリシスロスのようなエネルギー散逸により亀裂進展が抑制されるという Andrews モデルが提案されている\cite{andrews}。
また、ゴム系材料の破壊において粘弾性挙動として時間温度換算則が大変形を伴う破壊挙動にも成立し、室温では容易に破断する SBR がガラス転移温度に近い低温での伸長では高い伸びと強度を示すことも報告されている~\cite{smith}。

\subsection{``Phantom Network Model'' の確認}
``Phantom Network Model: PNM''


ゴム弾性の古典的なモデルである ``Affine Network Model'' からの発展形として、結節点の揺らぎに注目した``Phantom Network Model: PNM''が提案され、Flory によればメルト状態と同一なストランドのゆらぎを有するランダムネットワークにおいて PNM のふるまいを示すとされている~\cite{flory}。
我々は、この結節点のゆらぎ由来の散逸が、分子鎖描像のようなミクロなスケールでの粘弾性的なエネルギー散逸モデルとなりうるのではないかと考え、これまで検討を進めている。


以前に、規則構造ネットワークをベースとしてユニットセル間における規則性をランダムへと変えることで架橋欠損のないネットワークを作成して PNM を再現できることを報告した~\cite{sasaki}。



本報告では、ランダムな接続性を有するネットワークポリマーの緩和挙動について、MD シミュレーションにより検討した結果について報告する。






近年、ソフトマターの階層的な構造設計の考え方が深化し、力学特性に優れたネットワークポリマーの材料設計にも応用されている。
旧知の材料であるゴムの大きな破壊靭性の由来については、 ヒステリシスロスのようなエネルギー散逸により亀裂進展が抑制されるという Andrews モデルが提案されている\cite{andrews}。
また、ゴム系材料の破壊において粘弾性挙動として時間温度換算則が大変形を伴う破壊挙動にも成立し、室温では容易に破断する SBR がガラス転移温度に近い低温での伸長では高い伸びと強度を示すことも報告されている~\cite{smith}。

ゴム弾性の古典的なモデルである ``Affine Network Model'' からの発展形として、結節点の揺らぎに注目した``Phantom Network Model: PNM''が提案され、Flory によればメルト状態と同一なストランドのゆらぎを有するランダムネットワークにおいて PNM のふるまいを示すとされている~\cite{flory}。
我々は、この結節点のゆらぎ由来の散逸が、分子鎖描像のようなミクロなスケールでの粘弾性的なエネルギー散逸モデルとなりうるのではないかと考え、これまで検討を進めている。
以前に、規則構造ネットワークをベースとしてユニットセル間における規則性をランダムへと変えることで架橋欠損のないネットワークを作成して PNM を再現できることを報告した~\cite{sasaki}。
本報告では、ランダムな接続性を有するネットワークポリマーの緩和挙動について、MD シミュレーションにより検討した結果について報告する。






\vspace{-1mm}
\section{結果}
\subsection{シミュレーションについて}
既報~\cite{sasaki}に従い、ランダムな接続性を有する 4 および 3 分岐のネットワークを作成し、その平衡状態および変形(一軸伸張およびずりせん断)時の振る舞いについて、OCTA 上の COGNAC シミュレーターを用いた分子動力学シミュレーションにより評価した。


% \subsection{ネットワークの線形粘弾性}
% 変形速度を変化させた場合の SS カーブを Fig.\ref{fig:stretch} に示した。
% 平衡状態の MD シミュレーションから Green-Kubo 公式により求めたネットワークの応力緩和関数(赤線)および $\nu k_B T$ から算出したゴム弾性プラトーの値を緩和関数から差し引いたもの(緑線)を併せて Fig.\ref{fig: Relux} に示した。
% また、ストランドと同等な自由鎖(N=46)のラウス緩和(最長ラウス緩和時間 $\tau_R = 2700$)についても示している。

% この類似性から、少なくとも、今回検討した単純な規則構造を有するネットワークにおいては、架橋構造の緩和時間への寄与は少ないものと推定できた。

% % \begin{wrapfigure}{r}{65mm}
% % %\vspace{-1\baselineskip}
% % 	\begin{center}
% % 	\includegraphics[width=60mm]{./fig/N44_rev_SS.pdf}
% % 	\caption{Hysteresis Curves from different elongation position}
% % 	\label{fig: hyst}
% % 	\end{center}
% % \vspace{-5mm}
% % \end{wrapfigure}

\subsection{力学応答の評価}
4 分岐のネットワークポリマーに対して、変形速度の異なるせん断変形(1e-2 $\sim$ 5e-5 $\lambda/\tau$)時の SS カーブを、各種モデルの理論曲線と共に Fig. \ref{fig:deform} に示した。
変形速度の低減により、$\lambda<1$ 程度の小さなひずみでは PNM に漸近していた。
PNM へと漸近する変形速度 (2e-4 $\lambda/\tau$) で周期的なせん断変形 ($\lambda = 1$) を付与した結果(Fig. \ref{fig:hyst})においても、複数回の変形に対しても迅速な回復を伴った力学的ヒステリシス (Hysteresis loss $\simeq$ 35\%) を示すことが確認できた。
% また、伸長速度を遅くすることにより、ヒステリシス強度が減少することも確認できた。

変形モードの違いによって上記の挙動が変化することも見出しており、その詳細についても報告予定である。

% \section{おわりに}

% 本報告においては、単純な規則構造を有するネットワークの線形緩和現象と任意の変形速度での力学応答との関係から力学的ヒステリシスが生じることを確認し、その発現メカニズムがストランドの緩和現象に起因するものであることを推定した。
% 実際の破壊現象はこれほど単純ではなく、大変形時の非線形応答を考慮する必要は大きいと考えている。
% さらなる検討を進めていきたい。


\begin{figure}[hb]
    \begin{minipage}{0.5\hsize}
        \begin{center}
        \includegraphics[width=.7\textwidth]{Shear_Random_4chain_N20.png}
        \caption{Stress-Strain Curves for 4-chain NW at varied shear rate (1e-2 $\sim$ 5e-5 $\lambda/\tau$)}
        \label{fig:deform}
        \end{center}
    \end{minipage}
    \begin{minipage}{0.5\hsize}
        \begin{center}
        \includegraphics[width=.7\textwidth]{CyclicDeform_4chain_rate_2e-4.png}
        \caption{Hysteresis Curves for 4-chain NW by Cyclic Shear ($\lambda = 1$): shear rate 2e-4 $\lambda/\tau$}
        \label{fig:hyst}
        \end{center}
    \end{minipage}
    % \begin{minipage}{0.33\hsize}
    %     \begin{center}
    %     \includegraphics[width=\textwidth]{hyst_4Chain.png}
    %     \caption{Strand Exchange Procedure}
    %     \label{fig:exc}
    %     \end{center}
    % \end{minipage}
\end{figure}

\vspace{-7mm}
\begin{thebibliography}{99}
    \bibitem{andrews} E. H. Andrews, Y. Fukahori Journal of Materials Science, 12, 1307 (1977)
    \bibitem{smith} T. L. Smith, R. A. Dickie Journal of Polymer Science Part A-2: Polymer Physics, 7, 635 (1969)
    \bibitem{flory} P. J. Flory Proceedings of the Royal Society of London. Series A, 351, 351 (1976)
    \bibitem{sasaki} 佐々木裕, 第69回レオロジー討論会 要旨集 (2021)
\end{thebibliography}

\end{document}