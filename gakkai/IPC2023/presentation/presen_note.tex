% \documentclass[dvipdfmx,12pt]{beamer}
% \usepackage{bxdpx-beamer}
% \usepackage{pxjahyper}
% % \usepackage[noto-otc]{pxchfon}
% \renewcommand{\kanjifamilydefault}{\gtdefault}
% \usepackage{pgfpages}
% \setbeamertemplate{note page}{\pagecolor{yellow!5}\vfill\insertnote\vfill}
% \setbeameroption{show notes on second screen=right}
% \usetheme{metropolis}
% \title{Beamerによるプレゼンテーションサンプル}
% \author{柴田充也}
% \institute{Ubuntu Japanse Team}
% \date{2020年12月2日}
% \begin{document}

% \begin{frame}
%   \titlepage
% \end{frame}

% \section{はじめに}
% \begin{frame}\frametitle{本日お伝えしたいこと}
%   \tableofcontents
% \end{frame}

% \section{重要なこと}
% \begin{frame}\frametitle{いっこめ}
%   \begin{itemize}[<+->]
%     \item 項目1
%     \item 項目2
%       \note[item]{項目2について熱く語る}
%     \item 項目3
%   \end{itemize}
% \end{frame}

% \begin{frame}\frametitle{にこめ}
%   \begin{enumerate}
%     \item 項目1
%     \item 項目2
%     \item 項目3
%   \end{enumerate}
% \end{frame}

% \section{そこまで重要でないこと}
% \begin{frame}\frametitle{さんこめ}
%   \begin{itemize}
%     \item 項目1
%     \item 項目2
%     \item 項目3
%   \end{itemize}
% \end{frame}

% \end{document}




\documentclass{beamer}
\usetheme{default}

%% 発表ノート用の設定
\usepackage{pgfpages}
%\setbeameroption{hide notes} % スライドのみ作成
%\setbeameroption{show only notes} % 発表ノートのみ作成
\setbeameroption{show notes on second screen=right} % スライドと発表ノートを横並びに作成
\setbeamertemplate{note page}{\pagecolor{pink!50}\insertnote}  % 発表ノートの設定 (ナビゲーションは非表示)

\title{Beamer: presenter notes}
\author{charmie11}
\date{}

\begin{document}
\begin{frame}[plain]
    \maketitle
    % 発表ノート
    \note{
        can write notes on title slide
    }
\end{frame}

\begin{frame}{Frame Title}
    \begin{itemize}
        \item foo
        \item bar
    \end{itemize}

    % 発表ノート
    \note{
        can write as normal frame
        \begin{enumerate}
            \item item 1
            \item item 2
        \end{enumerate}
    }
\end{frame}

\end{document}