\begin{itembox}[l]{Andrews 理論}
    \begin{columns}[totalwidth=.9\textwidth]
        \column{.75\textwidth}
            \begin{itemize}
                \item クラックの進展を\alert{抑制}
                \item Andrews 理論\cite{andrews}
                    \begin{itemize}
                        \item クラックの応力場
                        \item クラック進展時に、{\color{red} エネルギー散逸}
                        \item \alert{ヒステリシスに由来}
                    \end{itemize}	
            \end{itemize}
        \column{.24\textwidth}
            \includegraphics[width=.9\textwidth]{crack.png}     
    \end{columns}
\end{itembox}


% \begin{alertblock}{疲労破壊も考慮すると}
%     \begin{itemize}
%         \item \alert{可逆的}であることが望ましい。\textcolor{blue}{$\neq$ 犠牲結合}
%         \item 変形の周期に対応できるように、\alert{回復速度}も重要。
%     \end{itemize}
% \end{alertblock}